\HeaderA{matrixplot}{Matrix plot}{matrixplot}
\aliasA{iimagMiss}{matrixplot}{iimagMiss}
\aliasA{TKRmatrixplot}{matrixplot}{TKRmatrixplot}
\keyword{hplot}{matrixplot}
%
\begin{Description}\relax
Create a matrix plot, in which all cells of a data matrix are visualized by 
rectangles.  Available data is coded according to a continuous color scheme, 
while missing/imputed data is visualized by a clearly distinguishable color.
\end{Description}
%
\begin{Usage}
\begin{verbatim}
matrixplot(x, delimiter = NULL, sortby = NULL, col = c("red","orange"),
    gamma = 2.2, fixup = TRUE, xlim = NULL, ylim = NULL, main = NULL,
    sub = NULL, xlab = NULL, ylab = NULL, axes = TRUE, labels = axes,
    xpd = NULL, interactive = TRUE, ...)

TKRmatrixplot(x, ..., delimiter = NULL, hscale = NULL, vscale = NULL,
    TKRpar = list())

iimagMiss(x, delimiter = NULL, sortby = NULL, col = c("red","orange"),
    main = NULL, sub = NULL, xlab = NULL, ylab = NULL, xlim = NULL,
    ylim = NULL, axes = TRUE, xaxlabels = NULL, las = 3,
    interactive = TRUE, ...)
\end{verbatim}
\end{Usage}
%
\begin{Arguments}
\begin{ldescription}
\item[\code{x}] a matrix or \code{data.frame}.
\item[\code{delimiter}] a character-vector to distinguish between variables
and imputation-indices for imputed variables (therefore, \code{x} needs
to have \code{\LinkA{colnames}{colnames}}). If given, it is used to determine the
corresponding imputation-index for any imputed variable (a logical-vector
indicating which values of the variable have been imputed). If such
imputation-indices are found, they are used for highlighting and the
colors are adjusted	according to the given colors for imputed variables
(see \code{col}).
\item[\code{sortby}] a numeric or character value specifying the variable to sort 
the data matrix by, or \code{NULL} to plot without sorting.
\item[\code{col}] the colors to be used in the plot.  RGB colors may be specified as 
character strings or as objects of class 
"\code{\LinkA{RGB}{RGB}}".  HCL colors need to be specified as 
objects of class "\code{\LinkA{polarLUV}{polarLUV}}".  If only 
one color is supplied, it is used for missing and imputed data and a greyscale is used 
for available data. If two colors are supplied, the first is used for missing and
the second for imputed data and a greyscale for available data.
If three colors are supplied, the first is used as end 
color for the available data, while the start color is taken to be 
transparent for RGB or white for HCL.  Missing/imputed data is visualized by the 
second/third color in this case.  If four colors are supplied, the first is used 
as start color and the second as end color for the available data, while 
the third/fourth color is used for missing/imputed data.
\item[\code{gamma}] numeric; the display \emph{gamma} value (see 
\code{\LinkA{hex}{hex}}).
\item[\code{fixup}] a logical indicating whether the colors should be corrected to 
valid RGB values (see \code{\LinkA{hex}{hex}}).
\item[\code{xlim, ylim}] axis limits.
\item[\code{main, sub}] main and sub title.
\item[\code{xlab, ylab}] axis labels.
\item[\code{axes}] a logical indicating whether axes should be drawn on the plot.
\item[\code{labels}] either a logical indicating whether labels should be plotted 
below each column, or a character vector giving the labels.
\item[\code{xpd}] a logical indicating whether the rectangles should be allowed to 
go outside the plot region.  If \code{NULL}, it defaults to \code{TRUE} 
unless axis limits are specified.
\item[\code{interactive}] a logical indicating whether a variable to be used for 
sorting can be selected interactively (see `Details').
\item[\code{xaxlabels}] a character vector containing the labels for the columns.  
If \code{NULL}, the column names of \code{x} will be used.
\item[\code{las}] the style of axis labels (see \code{\LinkA{par}{par}}).
\item[\code{...}] for \code{matrixplot} and \code{iimagMiss}, further graphical 
parameters to be passed to \code{\LinkA{plot.window}{plot.window}}, 
\code{\LinkA{title}{title}} and \code{\LinkA{axis}{axis}}.  For 
\code{TKRmatrixplot}, further arguments to be passed to \code{matrixplot}.
\item[\code{hscale}] horizontal scale factor for plot to be embedded in a 
\emph{Tcl/Tk} window (see `Details').  The default value depends on 
the number of variables.
\item[\code{vscale}] vertical scale factor for the plot to be embedded in a 
\emph{Tcl/Tk} window (see `Details').  The default value depends on 
the number of observations.
\item[\code{TKRpar}] a list of graphical parameters to be set for the plot to be 
embedded in a \emph{Tcl/Tk} window (see `Details' and 
\code{\LinkA{par}{par}}).
\end{ldescription}
\end{Arguments}
%
\begin{Details}\relax
In a \emph{matrix plot}, all cells of a data matrix are visualized by 
rectangles.  Available data is coded according to a continuous color scheme.  
To compute the colors via interpolation, the variables are first scaled to 
the interval \eqn{$[0,1]$}{}. Missing/imputed values can then be visualized by a 
clearly distinguishable color. It is thereby possible to use colors in the
\emph{HCL} or \emph{RGB} color space. A simple way of visualizing the 
magnitude of the available data is to apply a greyscale, which has the 
advantage that missing/imputed values can easily be distinguished by using a color 
such as red/orange.  Note that \code{-Inf} and \code{Inf} are always assigned the 
begin and end color, respectively, of the continuous color scheme.

Additionally, the observations can be sorted by the magnitude of a selected 
variable.  If \code{interactive} is \code{TRUE}, clicking in a column redraws 
the plot with observations sorted by the corresponding variable.  Clicking 
anywhere outside the plot region quits the interactive session.

\code{TKRmatrixplot} behaves like \code{matrixplot}, but uses 
\code{\LinkA{tkrplot}{tkrplot}} to embed the plot in a \emph{Tcl/Tk} window.  
This is useful if the number of observations and/or variables is large, 
because scrollbars allow to move from one part of the plot to another.
\end{Details}
%
\begin{Note}\relax
This is a much more powerful extension to the function  
\code{imagmiss} in the former CRAN package \code{dprep}.

\code{iimagMiss} is deprecated and may be omitted in future versions of 
\code{VIM}.  Use \code{matrixplot} instead.
\end{Note}
%
\begin{Author}\relax
Andreas Alfons, Matthias Templ, modifications by Bernd Prantner
\end{Author}
%
\begin{Examples}
\begin{ExampleCode}
data(sleep, package = "VIM")
## for missing values
x <- sleep[, -(8:10)]
x[,c(1,2,4,6,7)] <- log10(x[,c(1,2,4,6,7)])
matrixplot(x, sortby = "BrainWgt")

## for imputed values
x_imp <- kNN(sleep[, -(8:10)])
x_imp[,c(1,2,4,6,7)] <- log10(x_imp[,c(1,2,4,6,7)])
matrixplot(x_imp, delimiter = "_imp", sortby = "BrainWgt")
\end{ExampleCode}
\end{Examples}

\HeaderA{growdotMiss}{Growing dot map with information about missing/imputed values}{growdotMiss}
\aliasA{bubbleMiss}{growdotMiss}{bubbleMiss}
\keyword{hplot}{growdotMiss}
%
\begin{Description}\relax
Map with dots whose sizes correspond to the values in a certain
variable.  Observations with missing/imputed values in additional variables 
are highlighted.
\end{Description}
%
\begin{Usage}
\begin{verbatim}
growdotMiss(x, coords, map, pos = 1, delimiter = NULL,
    selection = c("any","all"), log = FALSE, col = c("skyblue","red",
    "skyblue4","red4","orange","orange4"), border = par("bg"),
    alpha = NULL, scale = NULL, size = NULL, exp = c(0, 0.95, 0.05),
    col.map = grey(0.5), legend = TRUE, legtitle = "Legend",
    cex.legtitle = par("cex"), cex.legtext = par("cex"), ncircles = 6,
    ndigits = 1, interactive = TRUE, ...)

bubbleMiss(...)
\end{verbatim}
\end{Usage}
%
\begin{Arguments}
\begin{ldescription}
\item[\code{x}] a vector, matrix or \code{data.frame}.
\item[\code{coords}] a matrix or \code{data.frame} with two columns giving 
the spatial coordinates of the observations.
\item[\code{map}] a background map to be passed to \code{\LinkA{bgmap}{bgmap}}.
\item[\code{pos}] a numeric value giving the index of the variable 
determining the dot sizes.
\item[\code{delimiter}] a character-vector to distinguish between variables
and imputation-indices for imputed variables (therefore, \code{x} needs
to have \code{\LinkA{colnames}{colnames}}). If given, it is used to determine the corresponding
imputation-index for any imputed variable (a logical-vector indicating
which values of the variable have been imputed). If such imputation-indices
are found, they are used for highlighting and the colors are adjusted 
according to the given colors for imputed variables (see \code{col}).
\item[\code{selection}] the selection method for highlighting missing/imputed values 
in multiple additional variables.  Possible values are \code{"any"} 
(highlighting of missing/imputed values in \emph{any} of the additional 
variables) and \code{"all"} (highlighting of missing/imputed values in 
\emph{all} of the additional variables).
\item[\code{log}] a logical indicating whether the variable given by 
\code{pos} should be log-transformed.
\item[\code{col}] a vector of length six giving the colors to be used in the 
plot.  If only one color is supplied, it is used for the borders of 
non-highlighted dots and the surface area of highlighted dots.  Else if 
two colors are supplied, they are recycled.
\item[\code{border}] a vector of length four giving the colors to be used for the 
borders of the growing dots.  Use \code{NA} to omit borders.
\item[\code{alpha}] a numeric value between 0 and 1 giving the level of 
transparency of the colors, or \code{NULL}.  This can be used to
prevent overplotting.
\item[\code{scale}] scaling factor of the map.
\item[\code{size}] a vector of length two giving the sizes for the smallest 
and largest dots.
\item[\code{exp}] a vector of length three giving the factors that define the 
shape of the exponential function (see `Details').
\item[\code{col.map}] the color to be used for the background map.
\item[\code{legend}] a logical indicating whether a legend should be plotted.
\item[\code{legtitle}] the title for the legend.
\item[\code{cex.legtitle}] the character expansion factor to be used for the 
title of the legend.
\item[\code{cex.legtext}] the character expansion factor to be used in the 
legend.
\item[\code{ncircles}] the number of circles displayed in the legend.
\item[\code{ndigits}] the number of digits displayed in the legend.  Note that \bsl{}
this is just a suggestion (see \code{\LinkA{format}{format}}).
\item[\code{interactive}] a logical indicating whether information about 
certain observations can be displayed interactively (see 
`Details').
\item[\code{...}] for \code{growdotMiss}, further arguments and graphical 
parameters to be passed to \code{\LinkA{bgmap}{bgmap}}.  For \code{bubbleMiss}, 
the arguments to be passed to \code{growdotMiss}.
\end{ldescription}
\end{Arguments}
%
\begin{Details}\relax
The smallest dots correspond to the 10\% quantile and the largest 
dots to the 99\% quantile.  In between, the dots grow exponentially, 
with \code{exp} defining the shape of the exponential function.
Missings/imputed missings in the variable of interest will be drawn as
rectangles.

If \code{interactive=TRUE}, detailed information for an observation 
can be printed on the console by clicking on the corresponding point.  
Clicking in a region that does not contain any points quits the 
interactive session.
\end{Details}
%
\begin{Note}\relax
The function was renamed to \code{growdotMiss} in version 1.3. 
\code{bubbleMiss} is a (deprecated) wrapper for \code{growdotMiss} 
for back compatibility with older versions. However, due to extended 
functionality, some of the argument positions have changed. 

The code is based on \code{\LinkA{bubbleFIN}{bubbleFIN}} from package 
\code{StatDA}.
\end{Note}
%
\begin{Author}\relax
Andreas Alfons, Bernd Prantner
\end{Author}
%
\begin{SeeAlso}\relax
\code{\LinkA{bgmap}{bgmap}}, \code{\LinkA{mapMiss}{mapMiss}}, 
\code{\LinkA{colormapMiss}{colormapMiss}}
\end{SeeAlso}
%
\begin{Examples}
\begin{ExampleCode}
data(chorizonDL, package = "VIM")
data(kola.background, package = "VIM")
coo <- chorizonDL[, c("XCOO", "YCOO")]
## for missing values
x <- chorizonDL[, c("Ca","As", "Bi")]
growdotMiss(x, coo, kola.background, border = "white")

## for imputed values
x_imp <- kNN(chorizonDL[,c("Ca","As","Bi" )])
growdotMiss(x_imp, coo, kola.background, delimiter = "_imp", border = "white")
\end{ExampleCode}
\end{Examples}

\HeaderA{vmGUImenu}{GUI for Visualization and Imputation of Missing Values}{vmGUImenu}
\keyword{multivariate}{vmGUImenu}
\keyword{hplot}{vmGUImenu}
%
\begin{Description}\relax
Graphical user interface for visualization and imputation of 
missing values.
\end{Description}
%
\begin{Usage}
\begin{verbatim}
vmGUImenu()
\end{verbatim}
\end{Usage}
%
\begin{Details}\relax
The \emph{Data} menu allows to select a data set from the \R{} workspace or 
load data into the workspace from \code{RData} files.  Furthermore, it can 
be used to transform variables, which are then appended to the data set
in use.  Commonly used transformations in official statistics are 
available, e.g., the Box-Cox transformation and the log-transformation 
as an important special case of the Box-Cox transformation.  In addition, 
several other transformations that are frequently used for compositional 
data are implemented. Background maps and coordinates for spatial data 
can be selected in the data menu as well.

After a data set was chosen, variables can be selected in the main menu, 
along with a method for scaling.  An important feature is that the 
variables will be used in the same order as they were selected, which 
is especially useful for parallel coordinate plots.  Variables for 
highlighting are distinguished from the plot variables and can be 
selected separately.  For more than one variable chosen for highlighting, 
it is possible to select whether observations with missing values in any 
or in all of these variables should be highlighted.

A plot method can be selected from the \emph{Visualization} menu. Note that
plots that are not applicable to the selected variables are disabled, for
example, if only one plot variable is selected, multivariate plots cannot 
be chosen.

The \emph{Imputation} menu offers robust imputation methods to impute
variables of the data set.

The \emph{Diagnostics} menu is similar to the \emph{Visualization} menu,
but is designed to verify the results after the imputation of missing 
values. 

Last, but not least, the \emph{Options} menu allows to set the colors, 
alpha channel and the delimiter for imputed variables to be used in the plots.
In addition, it contains an option to embed multivariate plots in
\code{Tcl/Tk} windows. This is useful if the number of observations and/or
variables is large, because scrollbars allow to move from one part of the
plot to another.

The section \emph{Imputation} is not yet implemented.
\end{Details}
%
\begin{Author}\relax
Andreas Alfons, based on an initial design by Matthias Templ,
modifications by Bernd Prantner
\end{Author}

\HeaderA{mapMiss}{Map with information about missing/imputed values}{mapMiss}
\keyword{hplot}{mapMiss}
%
\begin{Description}\relax
Map of observed and missing/imputed values.
\end{Description}
%
\begin{Usage}
\begin{verbatim}
mapMiss(x, coords, map, delimiter = NULL, selection = c("any","all"), 
    col = c("skyblue","red","orange"), alpha = NULL, pch = c(19,15),
    col.map = grey(0.5), legend = TRUE, interactive = TRUE, ...)
\end{verbatim}
\end{Usage}
%
\begin{Arguments}
\begin{ldescription}
\item[\code{x}] a vector, matrix or \code{data.frame}.
\item[\code{coords}] a \code{data.frame} or matrix with two columns giving 
the spatial coordinates of the observations.
\item[\code{map}] a background map to be passed to \code{\LinkA{bgmap}{bgmap}}.
\item[\code{delimiter}] a character-vector to distinguish between variables
and imputation-indices for imputed variables (therefore, \code{x} needs
to have \code{\LinkA{colnames}{colnames}}). If given, it is used to determine the corresponding
imputation-index for any imputed variable (a logical-vector indicating
which values of the variable have been imputed). If such imputation-indices
are found, they are used for highlighting and the colors are adjusted 
according to the given colors for imputed variables (see \code{col}).
\item[\code{selection}] the selection method for displaying missing/imputed values in 
the map.  Possible values are \code{"any"} (display missing/imputed values in 
\emph{any} variable) and \code{"all"} (display missing/imputed values in 
\emph{all} variables).
\item[\code{col}] a vector of length three giving the colors to be used for 
observed, missing and imputed values.  If a single color is supplied, it 
is used for all values.
\item[\code{alpha}] a numeric value between 0 and 1 giving the level of 
transparency of the colors, or \code{NULL}.  This can be used to
prevent overplotting.
\item[\code{pch}] a vector of length two giving the plot characters to be used 
for observed and missing/imputed values.  If a single plot character is 
supplied, it will be used for both.
\item[\code{col.map}] the color to be used for the background map.
\item[\code{legend}] a logical indicating whether a legend should be plotted.
\item[\code{interactive}] a logical indicating whether information about 
selected observations can be displayed interactively (see 
`Details').
\item[\code{...}] further graphical parameters to be passed to 
\code{\LinkA{bgmap}{bgmap}} and \code{\LinkA{points}{points}}.
\end{ldescription}
\end{Arguments}
%
\begin{Details}\relax
If \code{interactive=TRUE}, detailed information for an observation 
can be printed on the console by clicking on the corresponding point.  
Clicking in a region that does not contain any points quits the 
interactive session.
\end{Details}
%
\begin{Author}\relax
Matthias Templ, Andreas Alfons, modifications by Bernd Prantner
\end{Author}
%
\begin{SeeAlso}\relax
\code{\LinkA{bgmap}{bgmap}}, \code{\LinkA{bubbleMiss}{bubbleMiss}}, 
\code{\LinkA{colormapMiss}{colormapMiss}}
\end{SeeAlso}
%
\begin{Examples}
\begin{ExampleCode}
data(chorizonDL, package = "VIM")
data(kola.background, package = "VIM")
coo <- chorizonDL[, c("XCOO", "YCOO")]
## for missing values
x <- chorizonDL[, c("As", "Bi")]
mapMiss(x, coo, kola.background)

## for imputed values
x_imp <- kNN(chorizonDL[, c("As", "Bi")])
mapMiss(x_imp, coo, kola.background, delimiter = "_imp")
\end{ExampleCode}
\end{Examples}

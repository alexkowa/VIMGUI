\HeaderA{histMiss}{Histogram with information about missing/imputed values}{histMiss}
\keyword{hplot}{histMiss}
%
\begin{Description}\relax
Histogram with highlighting of missing/imputed values in other variables by 
splitting each bin into two parts.  Additionally, information about 
missing/imputed values in the variable of interest is shown on the right hand 
side.
\end{Description}
%
\begin{Usage}
\begin{verbatim}
histMiss(x, delimiter = NULL, pos = 1, selection = c("any","all"),
    breaks = "Sturges", right = TRUE, col = c("skyblue","red",
    "skyblue4","red4","orange","orange4"), border = NULL,
    main = NULL, sub = NULL, xlab = NULL, ylab = NULL, axes = TRUE,
    only.miss = TRUE, miss.labels = axes, interactive = TRUE, ...)
\end{verbatim}
\end{Usage}
%
\begin{Arguments}
\begin{ldescription}
\item[\code{x}] a vector, matrix or \code{data.frame}.
\item[\code{delimiter}] a character-vector to distinguish between variables
and imputation-indices for imputed variables (therefore, \code{x} needs
to have \code{\LinkA{colnames}{colnames}}). If given, it is used to determine the
corresponding imputation-index for any imputed variable (a logical-vector
indicating which values of the variable have been imputed). If such
imputation-indices are found, they are used for highlighting and the
colors are adjusted	according to the given colors for imputed variables
(see \code{col}).
\item[\code{pos}] a numeric value giving the index of the variable of 
interest.  Additional variables in \code{x} are used for 
highlighting.
\item[\code{selection}] the selection method for highlighting missing/imputed values 
in multiple additional variables.  Possible values are \code{"any"} 
(highlighting of missing/imputed values in \emph{any} of the additional 
variables) and \code{"all"} (highlighting of missing/imputed values in 
\emph{all} of the additional variables).
\item[\code{breaks}] either a character string naming an algorithm to compute 
the breakpoints (see \code{\LinkA{hist}{hist}}), or a numeric value giving 
the number of cells.
\item[\code{right}] logical; if \code{TRUE}, the histogram cells are 
right-closed (left-open) intervals.
\item[\code{col}] a vector of length six giving the colors to be used. If 
only one color is supplied, the bars are transparent and the 
supplied color is used for highlighting missing/imputed values.
Else if two colors are supplied, they are recycled.
\item[\code{border}] the color to be used for the border of the cells. 
Use \code{border=NA} to omit borders.
\item[\code{main, sub}] main and sub title.
\item[\code{xlab, ylab}] axis labels.
\item[\code{axes}] a logical indicating whether axes should be drawn 
on the plot.
\item[\code{only.miss}] logical; if \code{TRUE}, the missing/imputed values in the first 
variable are visualized by a single bar.  Otherwise, a small barplot 
is drawn on the right hand side (see `Details').
\item[\code{miss.labels}] either a logical indicating whether label(s) should 
be plotted below the bar(s) on the right hand side, or a character 
string or vector giving the label(s) (see `Details').
\item[\code{interactive}] a logical indicating whether the variables can be 
switched interactively (see `Details').
\item[\code{...}] further graphical parameters to be passed to 
\code{\LinkA{title}{title}} and \code{\LinkA{axis}{axis}}.
\end{ldescription}
\end{Arguments}
%
\begin{Details}\relax
If more than one variable is supplied, the bins for the variable of 
interest will be split according to missingness/number of imputed missings in the additional 
variables.

If \code{only.miss=TRUE}, the missing/imputed values in the variable of interest 
are visualized by one bar on the right hand side.  If additional variables 
are supplied, this bar is again split into two parts according to 
missingness/number of imputed missings in the additional variables.

Otherwise, a small barplot consisting of two bars is drawn on the right 
hand side.  The first bar corresponds to observed values in the variable 
of interest and the second bar to missing/imputed values.  Since these two bars are 
not on the same scale as the main barplot, a second y-axis is 
plotted on the right (if \code{axes=TRUE}).  Each of the two bars are 
again split into two parts according to missingness/number of imputed missings in the additional 
variables.  Note that this display does not make sense if only one 
variable is supplied, therefore \code{only.miss} is ignored in that case.

If \code{interactive=TRUE}, clicking in the left margin of the plot 
results in switching to the previous variable and clicking in the right 
margin results in switching to the next variable.  Clicking anywhere 
else on the graphics device quits the interactive session.  When 
switching to a categorical variable, a barplot is produced rather than 
a histogram.
\end{Details}
%
\begin{Value}
a list with the following components:
\begin{ldescription}
\item[\code{breaks}] the breakpoints.
\item[\code{counts}] the number of observations in each cell.
\item[\code{missings}] the number of highlighted observations in each cell.
\item[\code{mids}] the cell midpoints.
\end{ldescription}
\end{Value}
%
\begin{Note}\relax
Some of the argument names and positions have changed with version 1.3 
due to extended functionality and for more consistency with other plot 
functions in \code{VIM}.  For back compatibility, the arguments 
\code{axisnames} and \code{names.miss} can still be supplied to 
\code{...} and are handled correctly.  Nevertheless, they are 
deprecated and no longer documented.  Use \code{miss.labels} instead.
\end{Note}
%
\begin{Author}\relax
Andreas Alfons, Matthias Templ, modifications by Bernd Prantner
\end{Author}
%
\begin{SeeAlso}\relax
\code{\LinkA{spineMiss}{spineMiss}}, \code{\LinkA{barMiss}{barMiss}}
\end{SeeAlso}
%
\begin{Examples}
\begin{ExampleCode}
data(tao, package = "VIM")
## for missing values
x <- tao[, c("Air.Temp", "Humidity")]
histMiss(x)
histMiss(x, only.miss = FALSE)

## for imputed values
x_IMPUTED <- kNN(tao[, c("Air.Temp", "Humidity")])
histMiss(x_IMPUTED, delimiter = "_imp")
histMiss(x_IMPUTED, delimiter = "_imp", only.miss = FALSE)
\end{ExampleCode}
\end{Examples}

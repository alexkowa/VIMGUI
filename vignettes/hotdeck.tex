\HeaderA{hotdeck}{Hot-Deck Imputation}{hotdeck}
\keyword{manip}{hotdeck}
%
\begin{Description}\relax
Implementation of the popular Sequential, Random (within a domain) hot-deck algorithm for imputation.
\end{Description}
%
\begin{Usage}
\begin{verbatim}
hotdeck(data, variable = colnames(data), ord_var = NULL,
    domain_var = NULL, makeNA = NULL, NAcond = NULL, impNA = TRUE,
    donorcond = NULL, imp_var = TRUE, imp_suffix = "imp")
\end{verbatim}
\end{Usage}
%
\begin{Arguments}
\begin{ldescription}
\item[\code{data}] 
data.frame or matrix

\item[\code{variable}] 
variables where missing values should be imputed

\item[\code{ord\_var}] 
variables for sorting the data set before imputation

\item[\code{domain\_var}] 
variables for building domains and impute within these domains

\item[\code{makeNA}] 
vector of values, that should be converted to NA

\item[\code{NAcond}] 
a condition for imputing a NA

\item[\code{impNA}] 
TRUE/FALSE whether NA should be imputed

\item[\code{donorcond}] 
condition for the donors e.g. ">5"

\item[\code{imp\_var}] 
TRUE/FALSE if a TRUE/FALSE variables for each imputed variable should be created show the imputation status

\item[\code{imp\_suffix}] 
suffix for the TRUE/FALSE variables showing the imputation status

\end{ldescription}
\end{Arguments}
%
\begin{Value}
the imputed data set.
\end{Value}
%
\begin{Author}\relax
Alexander Kowarik
\end{Author}
%
\begin{Examples}
\begin{ExampleCode}
data(sleep)
sleepI <- hotdeck(sleep)
sleepI2 <- hotdeck(sleep,ord_var="BodyWgt",domain_var="Pred")
\end{ExampleCode}
\end{Examples}

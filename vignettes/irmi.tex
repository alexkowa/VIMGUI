\HeaderA{irmi}{Iterative robust model-based imputation (IRMI) }{irmi}
\keyword{manip}{irmi}
%
\begin{Description}\relax
In each step of the iteration,
one variable is used as a response variable and the remaining variables
serve as the regressors.
\end{Description}
%
\begin{Usage}
\begin{verbatim}
irmi(x, eps = 5, maxit = 100, mixed = NULL, count = NULL, step = FALSE, 
    robust = FALSE, takeAll = TRUE, noise = TRUE, noise.factor = 1,
    force = FALSE, robMethod = "MM", force.mixed = TRUE, mi = 1,
    addMixedFactors = FALSE, trace = FALSE)
\end{verbatim}
\end{Usage}
%
\begin{Arguments}
\begin{ldescription}
\item[\code{x}] 
data.frame or matrix

\item[\code{eps}] 
threshold for convergency

\item[\code{maxit}] 
maximum number of iterations

\item[\code{mixed}] 
column index of the semi-continuous variables

\item[\code{count}] 
column index of count variables


\item[\code{step}] 
a stepwise model selection is applied when the parameter is set to TRUE

\item[\code{robust}] 
if TRUE, robust regression methods will be applied

\item[\code{takeAll}] 
takes information of (initialised) missings in the response as well for regression imputation.

\item[\code{noise}] 
irmi has the option to add a random error term to the imputed values, this
creates the possibility for multiple imputation. The error term has mean 0 and
variance corresponding to the variance of the
regression residuals.

\item[\code{noise.factor}] 
amount of noise.

\item[\code{force}] 
if TRUE, the algorithm tries to find a solution in any case, possible by using different robust methods automatically.

\item[\code{robMethod}] 
regression method when the response is continuous.

\item[\code{force.mixed}] 
if TRUE, the algorithm tries to find a solution in any case, possible by using different robust methods automatically.

\item[\code{addMixedFactors
}] 
if factor variables for the mixed variables should be created for the regression models

\item[\code{mi}] 
number of multiple imputations.

\item[\code{trace}] 
Additional information about the iterations when trace equals TRUE.

\end{ldescription}
\end{Arguments}
%
\begin{Details}\relax
The method works sequentially and iterative. The method can deal with a mixture of continuous, semi-continuous, ordinal and nominal variables including 
outliers. 

A full description of the method will be uploaded soon in form of a package vignette.
\end{Details}
%
\begin{Value}
the imputed data set.
\end{Value}
%
\begin{Author}\relax
Matthias Templ, Alexander Kowarik
\end{Author}
%
\begin{SeeAlso}\relax
\code{\LinkA{mi}{mi}}
\end{SeeAlso}
%
\begin{Examples}
\begin{ExampleCode}
data(sleep)
irmi(sleep)
\end{ExampleCode}
\end{Examples}

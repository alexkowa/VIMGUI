\HeaderA{initialise}{Initialization of missing values}{initialise}
\keyword{manip}{initialise}
%
\begin{Description}\relax
Rough estimation of missing values in a vector according to its type.
\end{Description}
%
\begin{Usage}
\begin{verbatim}
initialise(x)
\end{verbatim}
\end{Usage}
%
\begin{Arguments}
\begin{ldescription}
\item[\code{x}] a vector.
\end{ldescription}
\end{Arguments}
%
\begin{Details}\relax
Missing values are imputed with the mean for vectors of class 
\code{"numeric"}, with the median for vectors of class \code{"integer"}, 
and with the mode for vectors of class \code{"factor"}.  Hence, \code{x} 
should be prepared in the following way: assign class \code{"numeric"} 
to numeric vectors, assign class \code{"integer"} to ordinal  vectors, 
and assign class \code{"factor"} to nominal or binary  vectors. 
\end{Details}
%
\begin{Value}
the initialized vector.
\end{Value}
%
\begin{Note}\relax
  
The function is used internally by some imputation algorithms.
\end{Note}
%
\begin{Author}\relax
Matthias Templ, modifications by Andreas Alfons
\end{Author}

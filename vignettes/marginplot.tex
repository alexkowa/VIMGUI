\HeaderA{marginplot}{Scatterplot with additional information in the margins}{marginplot}
\keyword{hplot}{marginplot}
%
\begin{Description}\relax
In addition to a standard scatterplot, information about missing/imputed
values is shown in the plot margins. Furthermore, imputed values are
highlighted in the scatterplot.
\end{Description}
%
\begin{Usage}
\begin{verbatim}
marginplot(x, delimiter = NULL, col = c("skyblue","red","red4","orange",
    "orange4"), alpha = NULL, pch = c(1,16), cex = par("cex"),
    numbers = TRUE, cex.numbers = par("cex"), zeros = FALSE, xlim = NULL,
    ylim = NULL, main = NULL, sub = NULL, xlab = NULL, ylab = NULL,
    ann = par("ann"), axes = TRUE, frame.plot = axes, ...)
\end{verbatim}
\end{Usage}
%
\begin{Arguments}
\begin{ldescription}
\item[\code{x}] a \code{matrix} or \code{data.frame} with two columns.
\item[\code{delimiter}] a character-vector to distinguish between variables
and imputation-indices for imputed variables (therefore, \code{x} needs
to have \code{\LinkA{colnames}{colnames}}). If given, it is used to determine the corresponding
imputation-index for any imputed variable (a logical-vector indicating
which values of the variable have been imputed). If such imputation-indices
are found, they are used for highlighting and the colors are adjusted 
according to the given colors for imputed variables (see \code{col}).
\item[\code{col}] a vector of length five giving the colors to be used in the plot.  
The first color is used for the scatterplot and the boxplots for the 
available data. In case of missing values, the second color is taken for
the univariate scatterplots and boxplots for missing values in one variable
and the third for the frequency of missing/imputed values in both variables
(see `Details'). Otherwise, in case of imputed values, the fourth
color is used for the highlighting, the frequency, the univariate
scatterplot and the boxplots of mputed values in the first variable and the
fifth color for the same applied to the second variable. A black color is
used for the highlighting and the frequency of imputed values in both
variables instead. If only one color is supplied, it is used for the
bivariate and univariate scatterplots and the boxplots for missing/imputed
values in one variable, whereas the boxplots for the available data are
transparent.  Else if two colors are supplied, the second one is recycled.
\item[\code{alpha}] a numeric value between 0 and 1 giving the level of transparency 
of the colors, or \code{NULL}.  This can be used to prevent overplotting.
\item[\code{pch}] a vector of length two giving the plot symbols to be used for the 
scatterplot and the univariate scatterplots.  If a single plot character is 
supplied, it is used for the scatterplot and the default value will be used 
for the univariate scatterplots (see `Details').
\item[\code{cex}] the character expansion factor to be used for the bivariate and 
univariate scatterplots.
\item[\code{numbers}] a logical indicating whether the frequencies of missing/imputed values 
should be displayed in the lower left of the plot (see `Details').
\item[\code{cex.numbers}] the character expansion factor to be used for the 
frequencies of the missing/imputed values.
\item[\code{zeros}] a logical vector of length two indicating whether the variables 
are semi-continuous, i.e., contain a considerable amount of zeros.  If 
\code{TRUE}, only the non-zero observations are used for drawing the 
respective boxplot.  If a single logical is supplied, it is recycled.
\item[\code{xlim, ylim}] axis limits.
\item[\code{main, sub}] main and sub title.
\item[\code{xlab, ylab}] axis labels.
\item[\code{ann}] a logical indicating whether plot annotation (\code{main}, 
\code{sub}, \code{xlab}, \code{ylab}) should be displayed.
\item[\code{axes}] a logical indicating whether both axes should be drawn on the 
plot.  Use graphical parameter \code{"xaxt"} or \code{"yaxt"} to suppress 
only one of the axes.
\item[\code{frame.plot}] a logical indicating whether a box should be drawn around 
the plot.
\item[\code{...}] further graphical parameters to be passed down (see 
\code{\LinkA{par}{par}}).
\end{ldescription}
\end{Arguments}
%
\begin{Details}\relax
Boxplots for available and missing/imputed data, as well as univariate scatterplots 
for missing/imputed values in one variable are shown in the plot margins. 

Imputed values in either of the variables are highlighted in the scatterplot.

Furthermore, the frequencies of the missing/imputed values can be displayed by a 
number (lower left of the plot). The number in the lower left corner is the 
number of observations that are missing/imputed in both variables.
\end{Details}
%
\begin{Note}\relax
Some of the argument names and positions have changed with versions 1.3 and 
1.4 due to extended functionality and for more consistency with other plot 
functions in \code{VIM}.  For back compatibility, the argument 
\code{cex.text} can still be supplied to \code{...} and is handled 
correctly.  Nevertheless, it is deprecated and no longer documented.  Use 
\code{cex.numbers} instead.
\end{Note}
%
\begin{Author}\relax
Andreas Alfons, Matthias Templ, modifications by Bernd Prantner
\end{Author}
%
\begin{SeeAlso}\relax
\code{\LinkA{scattMiss}{scattMiss}}
\end{SeeAlso}
%
\begin{Examples}
\begin{ExampleCode}

data(tao, package = "VIM")
data(chorizonDL, package = "VIM")
## for missing values
marginplot(tao[,c("Air.Temp", "Humidity")])
marginplot(log10(chorizonDL[,c("CaO", "Bi")]))

## for imputed values
marginplot(kNN(tao[,c("Air.Temp", "Humidity")]), delimiter = "_imp")
marginplot(kNN(log10(chorizonDL[,c("CaO", "Bi")])), delimiter = "_imp")

\end{ExampleCode}
\end{Examples}

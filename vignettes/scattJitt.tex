\HeaderA{scattJitt}{Bivariate jitter plot}{scattJitt}
\keyword{hplot}{scattJitt}
%
\begin{Description}\relax
Create a bivariate jitter plot.
\end{Description}
%
\begin{Usage}
\begin{verbatim}
scattJitt(x, delimiter = NULL, col = c("skyblue","red","red4","orange",
    "orange4"), alpha = NULL, cex = par("cex"), col.line = "lightgrey",
    lty = "dashed", lwd = par("lwd"), numbers = TRUE,
    cex.numbers = par("cex"), main = NULL, sub = NULL, xlab = NULL,
    ylab = NULL, axes = TRUE, frame.plot = axes,
    labels = c("observed","missing","imputed"), ...)
\end{verbatim}
\end{Usage}
%
\begin{Arguments}
\begin{ldescription}
\item[\code{x}] a \code{data.frame} or \code{matrix} with two columns.
\item[\code{delimiter}] a character-vector to distinguish between variables
and imputation-indices for imputed variables (therefore, \code{x} needs
to have \code{\LinkA{colnames}{colnames}}). If given, it is used to determine the
corresponding imputation-index for any imputed variable (a logical-vector
indicating which values of the variable have been imputed). If such
imputation-indices are found, they are used for highlighting and the
colors are adjusted	according to the given colors for imputed variables
(see \code{col}).
\item[\code{col}] a vector of length five giving the colors to be used in the 
plot.  The first color will be used for complete observations, the 
second/fourth color for missing/imputed values in only one variable,
and the third/fifth color for missing/imputed values in both variables.
If only one color is supplied, it is used for all.
Else if two colors are supplied, the second one is recycled.
\item[\code{alpha}] a numeric value between 0 and 1 giving the level of 
transparency of the colors, or \code{NULL}.  This can be used to
prevent overplotting.
\item[\code{cex}] the character expansion factor for the plot characters.
\item[\code{col.line}] the color for the lines dividing the plot region.
\item[\code{lty}] the line type for the lines dividing the plot region (see 
\code{\LinkA{par}{par}}).
\item[\code{lwd}] the line width for the lines dividing the plot region.
\item[\code{numbers}] a logical indicating whether the frequencies of 
observed and missing/imputed values should be displayed
(see `Details').
\item[\code{cex.numbers}] the character expansion factor to be used for the 
frequencies of the observed and missing/imputed values.
\item[\code{main, sub}] main and sub title.
\item[\code{xlab, ylab}] axis labels.
\item[\code{axes}] a logical indicating whether both axes should be drawn 
on the plot.  Use graphical parameter \code{"xaxt"} or \code{"yaxt"} 
to suppress just one of the axes.
\item[\code{frame.plot}] a logical indicating whether a box should be drawn 
around the plot.
\item[\code{labels}] a vector of length three giving the axis labels for the 
regions for observed, missing and imputed values (see `Details').
\item[\code{...}] further graphical parameters to be passed down (see 
\code{\LinkA{par}{par}}).
\end{ldescription}
\end{Arguments}
%
\begin{Details}\relax
The amount of observed and missing/imputed values is visualized by jittered points.  
Thereby the plot region is divided into up to four regions according to 
the existence of missing/imputed values in one or both variables.  In addition, 
the amount of observed and missing/imputed values can be represented by a number.
\end{Details}
%
\begin{Note}\relax
Some of the argument names and positions have changed with version 1.3 
due to extended functionality and for more consistency with other plot 
functions in \code{VIM}.  For back compatibility, the argument 
\code{cex.text} can still be supplied to \code{...} and is handled 
correctly.  Nevertheless, it is deprecated and no longer documented.  
Use \code{cex.numbers} instead.
\end{Note}
%
\begin{Author}\relax
Matthias Templ, modifications by Andreas Alfons and Bernd Prantner
\end{Author}
%
\begin{Examples}
\begin{ExampleCode}
data(tao, package = "VIM")
## for missing values
scattJitt(tao[, c("Air.Temp", "Humidity")])

## for imputed values
scattJitt(kNN(tao[, c("Air.Temp", "Humidity")]), delimiter = "_imp")
\end{ExampleCode}
\end{Examples}

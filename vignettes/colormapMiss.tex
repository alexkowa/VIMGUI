\HeaderA{colormapMiss}{Colored map with information about missing/imputed values}{colormapMiss}
\aliasA{colormapMissLegend}{colormapMiss}{colormapMissLegend}
\keyword{hplot}{colormapMiss}
%
\begin{Description}\relax
Colored map in which the proportion or amount of missing/imputed values in each 
region is coded according to a continuous or discrete color scheme.  The 
sequential color palette may thereby be computed in the \emph{HCL} or the 
\emph{RGB} color space.
\end{Description}
%
\begin{Usage}
\begin{verbatim}
colormapMiss(x, region, map, imp_index = NULL, prop = TRUE,
    polysRegion = 1:length(x), range = NULL, n = NULL, col = c("red",
    "orange"), gamma = 2.2, fixup = TRUE, coords = NULL, numbers = TRUE,
    digits = 2, cex.numbers = 0.8, col.numbers = par("fg"),
    legend = TRUE, interactive = TRUE, ...)

colormapMissLegend(xleft, ybottom, xright, ytop, cmap, n = 1000,
    horizontal = TRUE, digits = 2, cex.numbers = 0.8,
    col.numbers = par("fg"), ...)
\end{verbatim}
\end{Usage}
%
\begin{Arguments}
\begin{ldescription}
\item[\code{x}] a numeric vector.
\item[\code{region}] a vector or factor of the same length as \code{x} giving the 
regions.
\item[\code{map}] an object of any class that contains polygons and provides its own 
plot method (e.g., \code{"SpatialPolygons"} from package \code{sp}).
\item[\code{imp\_index}] a logical-vector indicating which values of `x' have
been imputed. If given, it is used for highlighting and	the colors are
adjusted according to the given colors for imputed variables (see \code{col}).
\item[\code{prop}] a logical indicating whether the proportion of missing/imputed values 
should be used rather than the total amount.
\item[\code{polysRegion}] a numeric vector specifying the region that each polygon 
belongs to.
\item[\code{range}] a numeric vector of length two specifying the range (minimum and 
maximum) of the proportion or amount of missing/imputed values to be used for the 
color scheme.
\item[\code{n}] for \code{colormapMiss}, the number of equally spaced cut-off points 
for a discretized color scheme.  If this is not a positive integer, a 
continuous color scheme is used (the default).  In the latter case, the 
number of rectangles to be drawn in the legend can be specified in 
\code{colormapMissLegend}.  A reasonably large number makes it appear 
continuously.
\item[\code{col}] the color range (start end end) to be used.  RGB colors may be 
specified as character strings or as objects of class 
"\code{\LinkA{RGB}{RGB}}".  HCL colors need to be specified as 
objects of class "\code{\LinkA{polarLUV}{polarLUV}}".  If only 
one color is supplied, it is used as end color, while the start color is 
taken to be transparent for RGB or white for HCL.
\item[\code{gamma}] numeric; the display \emph{gamma} value (see 
\code{\LinkA{hex}{hex}}).
\item[\code{fixup}] a logical indicating whether the colors should be corrected to 
valid RGB values (see \code{\LinkA{hex}{hex}}).
\item[\code{coords}] a matrix or \code{data.frame} with two columns giving the 
coordinates for the labels.
\item[\code{numbers}] a logical indicating whether the corresponding proportions or 
numbers of missing/imputed values should be used as labels for the regions.
\item[\code{digits}] the number of digits to be used in the labels (in case of 
proportions).
\item[\code{cex.numbers}] the character expansion factor to be used for the labels.
\item[\code{col.numbers}] the color to be used for the labels.
\item[\code{legend}] a logical indicating whether a legend should be plotted.
\item[\code{interactive}] a logical indicating whether more detailed information 
about missing/imputed values should be displayed interactively (see 
`Details').
\item[\code{xleft}] left \emph{x} position of the legend.
\item[\code{ybottom}] bottom \emph{y} position of the legend.
\item[\code{xright}] right \emph{x} position of the legend.
\item[\code{ytop}] top \emph{y} position of the legend.
\item[\code{cmap}] a list as returned by \code{colormapMiss} that contains the 
required information for the legend.
\item[\code{horizontal}] a logical indicating whether the legend should be drawn 
horizontally or vertically.
\item[\code{...}] further arguments to be passed to \code{plot}.
\end{ldescription}
\end{Arguments}
%
\begin{Details}\relax
The proportion or amount of missing/imputed values in \code{x} of each region is 
coded according to a continuous or discrete color scheme in the color range 
defined by \code{col}.  In addition, the proportions or numbers can be shown 
as labels in the regions.

If \code{interactive} is \code{TRUE}, clicking in a region displays more 
detailed information about missing/imputed values on the \R{} console.  Clicking 
outside the borders quits the interactive session.
\end{Details}
%
\begin{Value}
\code{colormapMiss} returns a list with the following components:
\begin{ldescription}
\item[\code{nmiss}] a numeric vector containing the number of missing/imputed values in each 
region.
\item[\code{nobs}] a numeric vector containing the number of observations in each 
region.
\item[\code{pmiss}] a numeric vector containing the proportion of missing values in 
each region.
\item[\code{prop}] a logical indicating whether the proportion of missing/imputed values 
have been used rather than the total amount.
\item[\code{range}] the range of the proportion or amount of missing/imputed values 
corresponding to the color range.
\item[\code{n}] either a positive integer giving the number of equally spaced 
cut-off points for a discretized color scheme, or \code{NULL} for a 
continuous color scheme.
\item[\code{start}] the start color of the color scheme.
\item[\code{end}] the end color of the color scheme.
\item[\code{space}] a character string giving the color space (either \code{"rgb"} 
for RGB colors or \code{"hcl"} for HCL colors).
\item[\code{gamma}] numeric; the display \emph{gamma} value (see 
\code{\LinkA{hex}{hex}}).
\item[\code{fixup}] a logical indicating whether the colors have been corrected to 
valid RGB values (see \code{\LinkA{hex}{hex}}).
\end{ldescription}
\end{Value}
%
\begin{Note}\relax
Some of the argument names and positions have changed with versions 1.3 and 
1.4 due to extended functionality and for more consistency with other plot 
functions in \code{VIM}.  For back compatibility, the arguments 
\code{cex.text} and \code{col.text} can still be supplied to \code{...} and 
are handled correctly.  Nevertheless, they are deprecated and no longer 
documented.  Use \code{cex.numbers} and \code{col.numbers} instead.
\end{Note}
%
\begin{Author}\relax
Andreas Alfons, modifications by Bernd Prantner
\end{Author}
%
\begin{SeeAlso}\relax
\code{\LinkA{colSequence}{colSequence}}, \code{\LinkA{growdotMiss}{growdotMiss}}, \code{\LinkA{mapMiss}{mapMiss}}
\end{SeeAlso}

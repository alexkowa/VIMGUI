\HeaderA{scattMiss}{Scatterplot with information about missing/imputed values}{scattMiss}
\keyword{hplot}{scattMiss}
%
\begin{Description}\relax
In addition to a standard scatterplot, lines are plotted for the missing
values in one variable. If there are imputed values, they will be
highlighted.
\end{Description}
%
\begin{Usage}
\begin{verbatim}
scattMiss(x, delimiter = NULL, side = 1, col = c("skyblue","red",
    "orange","lightgrey"), alpha = NULL, lty = c("dashed","dotted"),
    lwd = par("lwd"), quantiles = c(0.5, 0.975), inEllipse = FALSE,
    zeros = FALSE, xlim = NULL, ylim = NULL, main = NULL, sub = NULL,
    xlab = NULL, ylab = NULL, interactive = TRUE, ...)
\end{verbatim}
\end{Usage}
%
\begin{Arguments}
\begin{ldescription}
\item[\code{x}] a \code{matrix} or \code{data.frame} with two columns.
\item[\code{delimiter}] a character-vector to distinguish between variables
and imputation-indices for imputed variables (therefore, \code{x} needs
to have \code{\LinkA{colnames}{colnames}}). If given, it is used to determine the
corresponding imputation-index for any imputed variable (a logical-vector
indicating which values of the variable have been imputed). If such
imputation-indices are found, they are used for highlighting and the
colors are adjusted	according to the given colors for imputed variables
(see \code{col}).
\item[\code{side}] if \code{side=1}, a rug representation and vertical lines are 
plotted for the missing/imputed values in the second variable; if \code{side=2}, a 
rug representation and horizontal lines for the missing/imputed values in the first 
variable.
\item[\code{col}] a vector of length four giving the colors to be used in the plot.  
The first color is used for the scatterplot, the second/third color for the 
rug representation for missing/imputed values. The second color is also
used for the lines for missing values. Imputed values will be highlighted
with the third color, and the fourth color is used for the ellipses (see
`Details'). If only one color is supplied, it is used for the
scatterplot, the rug representation and the lines, whereas the default
color is used for the ellipses.  Else if a vector of length two is 
supplied, the default color is used for the ellipses as well.
\item[\code{alpha}] a numeric value between 0 and 1 giving the level of transparency 
of the colors, or \code{NULL}.  This can be used to prevent overplotting.
\item[\code{lty}] a vector of length two giving the line types for the lines and
ellipses.  If a single value is supplied, it will be used for both.
\item[\code{lwd}] a vector of length two giving the line widths for the lines and
ellipses.  If a single value is supplied, it will be used for both.
\item[\code{quantiles}] a vector giving the quantiles of the chi-square distribution 
to be used for the tolerance ellipses, or \code{NULL} to suppress plotting 
ellipses (see `Details').
\item[\code{inEllipse}] plot lines only inside the largest ellipse.  Ignored 
if \code{quantiles} is \code{NULL} or if there are imputed values.
\item[\code{zeros}] a logical vector of length two indicating whether the variables 
are semi-continuous, i.e., contain a considerable amount of zeros.  If 
\code{TRUE}, only the non-zero observations are used for computing the 
tolerance ellipses.  If a single logical is supplied, it is recycled.  
Ignored if \code{quantiles} is \code{NULL}.
\item[\code{xlim, ylim}] axis limits.
\item[\code{main, sub}] main and sub title.
\item[\code{xlab, ylab}] axis labels.
\item[\code{interactive}] a logical indicating whether the \code{side} argument can 
be changed interactively (see `Details').
\item[\code{...}] further graphical parameters to be passed down (see 
\code{\LinkA{par}{par}}).
\end{ldescription}
\end{Arguments}
%
\begin{Details}\relax
Information about missing values in one variable is included as vertical or 
horizontal lines, as determined by the \code{side} argument.  The lines are 
thereby drawn at the observed x- or y-value. In case of imputed values, they
will additionally be highlighted in the scatterplot. Supplementary, percentage
coverage ellipses can be drawn to give a clue about the shape of the
bivariate data distribution.

If \code{interactive}is \code{TRUE}, clicking in the bottom margin redraws 
the plot with information about missing/imputed values in the first variable and 
clicking in the left margin redraws the plot with information about missing/imputed 
values in the second variable.  Clicking anywhere else in the plot quits the 
interactive session.
\end{Details}
%
\begin{Note}\relax
The argument \code{zeros} has been introduced in version 1.4. As a result, 
some of the argument positions have changed.
\end{Note}
%
\begin{Author}\relax
Andreas Alfons, modifications by Bernd Prantner
\end{Author}
%
\begin{SeeAlso}\relax
\code{\LinkA{marginplot}{marginplot}}
\end{SeeAlso}
%
\begin{Examples}
\begin{ExampleCode}
data(tao, package = "VIM")
## for missing values
scattMiss(tao[,c("Air.Temp", "Humidity")])

## for imputed values
scattMiss(kNN(tao[,c("Air.Temp", "Humidity")]), delimiter = "_imp")
\end{ExampleCode}
\end{Examples}

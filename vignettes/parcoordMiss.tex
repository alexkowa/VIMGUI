\HeaderA{parcoordMiss}{Parallel coordinate plot with information about missing/imputed values}{parcoordMiss}
\aliasA{TKRparcoordMiss}{parcoordMiss}{TKRparcoordMiss}
\keyword{hplot}{parcoordMiss}
%
\begin{Description}\relax
Parallel coordinate plot with adjustments for missing/imputed values. Missing values 
in the plotted variables may be represented by a point above the 
corresponding coordinate axis to prevent disconnected lines. In addition, 
observations with missing/imputed values in selected variables may be highlighted.
\end{Description}
%
\begin{Usage}
\begin{verbatim}
parcoordMiss(x, delimiter = NULL, highlight = NULL,
    selection = c("any","all"), plotvars = NULL, plotNA = TRUE,
    col = c("skyblue","red","skyblue4","red4","orange","orange4"),
    alpha = NULL, lty = par("lty"), xlim = NULL, ylim = NULL,
    main = NULL, sub = NULL, xlab = NULL, ylab = NULL, labels = TRUE,
    xpd = NULL, interactive = TRUE, ...)

TKRparcoordMiss(x, delimiter = NULL, highlight = NULL,
    selection = c("any","all"), plotvars = NULL, plotNA = TRUE,
    col = c("skyblue","red","skyblue4","red4","orange","orange4"),
    alpha = NULL, ..., hscale = NULL, vscale = 1, TKRpar = list())
\end{verbatim}
\end{Usage}
%
\begin{Arguments}
\begin{ldescription}
\item[\code{x}] a matrix or \code{data.frame}.
\item[\code{delimiter}] a character-vector to distinguish between variables
and imputation-indices for imputed variables (therefore, \code{x} needs
to have \code{\LinkA{colnames}{colnames}}). If given, it is used to determine the corresponding
imputation-index for any imputed variable (a logical-vector indicating
which values of the variable have been imputed). If such imputation-indices
are found, they are used for highlighting and the colors are adjusted 
according to the given colors for imputed variables (see \code{col}).
\item[\code{highlight}] a vector giving the variables to be used for highlighting.  
If \code{NULL} (the default), all variables are used for highlighting.
\item[\code{selection}] the selection method for highlighting missing/imputed values in 
multiple highlight variables.  Possible values are \code{"any"} 
(highlighting of missing/imputed values in \emph{any} of the highlight variables) 
and \code{"all"} (highlighting of missing/imputed values in \emph{all} of the 
highlight variables).
\item[\code{plotvars}] a vector giving the variables to be plotted.  If \code{NULL} 
(the default), all variables are plotted.
\item[\code{col}] if \code{plotNA} is \code{TRUE}, a vector of length six giving 
the colors to be used for observations with different combinations of 
observed and missing/imputed values in the plot variables and highlight variables 
(vectors of length one or two are recycled).  Otherwise, a vector of length 
two giving the colors for non-highlighted and highlighted observations (if 
a single color is supplied, it is used for both).
\item[\code{plotNA}] a logical indicating whether missing values in the plot 
variables should be represented by a point above the corresponding 
coordinate axis to prevent disconnected lines.
\item[\code{alpha}] a numeric value between 0 and 1 giving the level of transparency 
of the colors, or \code{NULL}.  This can be used to prevent overplotting.
\item[\code{lty}] if \code{plotNA} is \code{TRUE}, a vector of length four giving 
the line types to be used for observations with different combinations of 
observed and missing/imputed values in the plot variables and highlight variables 
(vectors of length one or two are recycled).  Otherwise, a vector of length 
two giving the line types for non-highlighted and highlighted observations 
(if a single line type is supplied, it is used for both).
\item[\code{xlim, ylim}] axis limits.
\item[\code{main, sub}] main and sub title.
\item[\code{xlab, ylab}] axis labels.
\item[\code{labels}] either a logical indicating whether labels should be plotted 
below each coordinate axis, or a character vector giving the labels.
\item[\code{xpd}] a logical indicating whether the lines should be allowed to go 
outside the plot region.  If \code{NULL}, it defaults to \code{TRUE} unless 
axis limits are specified.
\item[\code{interactive}] a logical indicating whether interactive features should 
be enabled (see `Details').
\item[\code{...}] for \code{parcoordMiss}, further graphical parameters to be 
passed down (see \code{\LinkA{par}{par}}).  For \code{TKRparcoordMiss}, 
further arguments to be passed to \code{parcoordMiss}.
\item[\code{hscale}] horizontal scale factor for plot to be embedded in a 
\emph{Tcl/Tk} window (see `Details').  The default value depends on 
the number of variables.
\item[\code{vscale}] vertical scale factor for the plot to be embedded in a 
\emph{Tcl/Tk} window (see `Details').
\item[\code{TKRpar}] a list of graphical parameters to be set for the plot to be 
embedded in a \emph{Tcl/Tk} window (see `Details' and 
\code{\LinkA{par}{par}}).
\end{ldescription}
\end{Arguments}
%
\begin{Details}\relax
In parallel coordinate plots, the variables are represented by parallel 
axes.  Each observation of the scaled data is shown as a line.  Observations 
with missing/imputed values in selected variables may thereby be highlighted.  
However, plotting variables with missing values results in disconnected 
lines, making it impossible to trace the respective observations across 
the graph.  As a remedy, missing values may be represented by a point above 
the corresponding coordinate axis, which is separated from the main plot by a 
small gap and a horizontal line, as determined by \code{plotNA}.  Connected 
lines can then be drawn for all observations.  Nevertheless, a caveat of this 
display is that it may draw attention away from the main relationships 
between the variables.

If \code{interactive} is \code{TRUE}, it is possible switch between this 
display and the standard display without the separate level for missing 
values by clicking in the top margin of the plot. In addition, the variables 
to be used for highlighting can be selected interactively.  Observations with 
missing/imputed values in any or in all of the selected variables are highlighted (as 
determined by \code{selection}).  A variable can be added to the selection by 
clicking on a coordinate axis.  If a variable is already selected, clicking 
on its coordinate axis removes it from the selection.  Clicking anywhere 
outside the plot region (except the top margin, if missing/imputed values exist) 
quits the interactive session.

\code{TKRparcoordMiss} behaves like \code{parcoordMiss}, but uses 
\code{\LinkA{tkrplot}{tkrplot}} to embed the plot in a \emph{Tcl/Tk} window.  
This is useful if the number of variables is large, because scrollbars allow 
to move from one part of the plot to another. 
\end{Details}
%
\begin{Note}\relax
Some of the argument names and positions have changed with versions 1.3 and 
1.4 due to extended functionality and for more consistency with other plot 
functions in \code{VIM}.  For back compatibility, the arguments 
\code{colcomb} and \code{xaxlabels} can still be supplied to \code{...} 
and are handled correctly.  Nevertheless, they are deprecated and no longer 
documented.  Use \code{highlight} and \code{labels} instead.
\end{Note}
%
\begin{Author}\relax
Andreas Alfons, Matthias Templ, modifications by Bernd Prantner
\end{Author}
%
\begin{References}\relax
Wegman, E. J. (1990) Hyperdimensional data analysis using parallel
coordinates. \emph{Journal of the American Statistical Association}
\bold{85 (411)}, 664--675.
\end{References}
%
\begin{SeeAlso}\relax
\code{\LinkA{pbox}{pbox}}
\end{SeeAlso}
%
\begin{Examples}
\begin{ExampleCode}
data(chorizonDL, package = "VIM")
## for missing values
parcoordMiss(chorizonDL[,c(15,101:110)], 
    plotvars=2:11, interactive = FALSE)
legend("top", col = c("skyblue", "red"), lwd = c(1,1), 
    legend = c("observed in Bi", "missing in Bi"))

## for imputed values
parcoordMiss(kNN(chorizonDL[,c(15,101:110)]), delimiter = "_imp" ,
    plotvars=2:11, interactive = FALSE)
legend("top", col = c("skyblue", "orange"), lwd = c(1,1), 
    legend = c("observed in Bi", "imputed in Bi"))
\end{ExampleCode}
\end{Examples}
